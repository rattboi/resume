% resume.tex
% vim:set ft=tex spell:

\documentclass[10pt,letterpaper]{article}
\usepackage[letterpaper,margin=1in]{geometry}
\usepackage[utf8]{inputenc}
\usepackage{mdwlist}
\usepackage[T1]{fontenc}
\usepackage{textcomp}
\usepackage{tgpagella}
\pagestyle{empty}
\setlength{\tabcolsep}{0em}

% indentsection style, used for sections that aren't already in lists
% that need indentation to the level of all text in the document
\newenvironment{indentsection}[1]%
{\begin{list}{}%
	{\setlength{\leftmargin}{#1}}%
	\item[]%
}
{\end{list}}

% opposite of above; bump a section back toward the left margin
\newenvironment{unindentsection}[1]%
{\begin{list}{}%
	{\setlength{\leftmargin}{-0.5#1}}%
	\item[]%
}
{\end{list}}

% format two pieces of text, one left aligned and one right aligned
\newcommand{\headerrow}[2]
{\begin{tabular*}{\linewidth}{l@{\extracolsep{\fill}}r}
	#1 &
	#2 \\
\end{tabular*}}

% make "C++" look pretty when used in text by touching up the plus signs
\newcommand{\CPP}
{C\nolinebreak[4]\hspace{-.05em}\raisebox{.22ex}{\footnotesize\bf ++}}

% and the actual content starts here
\begin{document}

\begin{center}
{\LARGE \textbf{Bradon Kanyid}}

11815 SW 3rd St.\ \ \textbullet
\ \ Beaverton, OR 97005 \ \ \textbullet
\ \ (360) 820-5112\\
bradon@kanyid.org\ \ \textbullet
\ \ www.kanyid.org
\end{center}

\hrule
\vspace{-0.4em}
\subsection*{Objective}
\begin{indentsection}{\parindent}
\begin{description*}
	\item Obtain an Embedded Systems internship in the Computer Engineering field.
\end{description*}
\end{indentsection}
\vspace{1em}

\hrule
\vspace{-0.4em}
\subsection*{Education}

\begin{itemize}
	\parskip=0.1em

	\item 
	\headerrow
                {\textbf{B.S. Computer Engineering}}
		{\textbf{Current GPA: 3.82}}
	\\
	\headerrow
		{\emph{Portland State University}}
		{\emph{Graduation June 2013}}
\end{itemize}
\vspace{1em}

\hrule
\vspace{-0.4em}
\subsection*{Experience}

\begin{itemize}
	\parskip=0.1em

	\item
	\headerrow
		{\textbf{Automation Engineer}}
		{\textbf{Silver Bay Seafoods, LLC.}}
	\\
	\headerrow
		{\emph{Summers 2009 -- 2012}}
		{\emph{Craig, AK}}
	\begin{itemize*}
    \item Wrote ladder logic for automating plant's sensors and actuators, such as conveyor belts, hydraulic rams, joysticks, and heat-sealers. Developed touchscreen Human Machine Interfaces and SCADA for monitoring and controlling the automation systems, data-collection middleware between automation systems and business software using .NET and SQL. Developed internal company website for remote observation and statistics in ASP.net. 
	\end{itemize*}

	\item
	\headerrow
		{\textbf{Tech Support Representative \& Internal Technician}}
		{\textbf{POS-X Inc.}}
	\\
	\headerrow
		{\emph{2007 -- 2009}}
		{\emph{Bellingham, WA}}
	\begin{itemize*}
    \item Remote support for POS-X products, managing trouble tickets, e-mail, phone support, hardware repairs, and mass computer assembly. Developed network-based automated burn-in and imaging system for new computers using PXElinux, BartPE, and Norton Ghost.
	\end{itemize*}

	\item
	\headerrow
		{\textbf{Embedded Programming Intern}}
		{\textbf{Pacific Northwest National Laboratories}}
	\\
	\headerrow
		{\emph{Summers 1999 -- 2001}}
		{\emph{Richland, WA}}
	\begin{itemize*}
    \item Developed data-logging temperature sensor for an array of mass spectrometers. Built using a combination of a custom PIC-based system and print server to handle translation of RS232-Ethernet traffic. Wrote device's firmware in PIC assembly, and desktop application in Visual Basic 6 to data-log and graph historical data and trends.
	\end{itemize*}

\end{itemize}
\vspace{1em}

\hrule
\vspace{-0.4em}
\subsection*{Core Technical Skills}

\begin{indentsection}{\parindent}
\hyphenpenalty=1000
\begin{description*}
	\item[Proficient Languages:]
    C, \CPP, Assembly (ARM, x86, z80, PIC, 68k), Verilog, Ladder Logic, Bash Scripting 
	\item[Familiar Languages:]
    Python, Visual Basic .NET, C\#, \LaTeX, SQL, Java, Clojure 
	\item[Software:]
    Platform Agnostic (Linux/Windows/Mac/BSD), Embedded GNU Tools (GCC, GDB, Redboot), Visual Studio for .NET, VCS (git, subversion), Xilinx ISE WebPACK
	\item[Hardware:]
    Digital Design, PCB Layout, FPGAs, SMD Soldering 
	\item[Automation:]
    Allen-Bradley PLC/PanelView/VFD Development, Rockwell RSLogix 5000/500, FactoryTalk Studio, Automation OPC 	
\end{description*}
\end{indentsection}
\vspace{1em}

\hrule
\begin{center}
{\emph{Honors and Projects Continue on Next Page}}

\end{center}

\newpage
\hrule
\vspace{-0.4em}
\subsection*{Honors Societies \& Volunteering}
\begin{itemize}
  \item
	\headerrow
		{\textbf{Eta Kappa Nu}}
		{\textbf{web.cecs.pdx.edu/\textasciitilde eta/}}
	\\
  IEEE Honors Society. Limited to top 25\% of Department.

  \item
	\headerrow
		{\textbf{Womprats Audio Synthesizer}}
		{\textbf{github.com/killerfriend/womprats}}
	\\
  Honorable Mention in Industry Design Practices course.
  
  \item
	\headerrow
		{\textbf{IEEE Student Store}}
		{\textbf{ieee.pdx.edu}}
	\\
  Volunteering includes 4 hours per week of desk duties, as well as occasional weekend store resupply.

  \item
  \headerrow
    {\textbf{Computer Action Team (CAT)}}
		{\textbf{cat.pdx.edu/thecat.html}}
	\\
  The CAT is a voluntary IT program for Portland State University's School of Engineering. Volunteer 4 hours weekly at the CAT front desk, helping students with computer and networking issues, as well as handling trouble tickets and maintaining student-related computer services. Test and develop new systems and services used by students and fellow CAT members. 

  \item
  \headerrow
    {\textbf{International Aerial Robotic Competition}}
		{\textbf{devel.avt.cecs.pdx.edu/projects/iarc}}
	\\
  Participated in developing an autonomous quadcopter (4-rotor helicopter). Contributed to integrating various sensors to the quadcopter, specifically an infrared range finder for flight height information. Wrote software for embedded LPCxpresso flight computer. Developed documentation for other parts of the low and high-level systems.
\end{itemize}
\vspace{1em}

\hrule
\vspace{-0.4em}
\subsection*{Major Projects}
\begin{itemize}
  \item 
  \headerrow 
  {\textbf{Processor Cache Simulator (Verilog)}}
  {\textbf{github.com/ekrause/0xBEEFA55}}
  \\
  I developed a simple cache simulator with a small team for the final project in a Microprocessor System Design course. It read in textfile sample trace data, and displayed cache hit/miss statistics. I predominantly developed the core logic for the N-way set associative data and instruction caches modules in Verilog.
  \item 
  \headerrow 
  {\textbf{ARM-based Audio Synthesizer (C)}}
  {\textbf{github.com/killerfriend/womprats}}
  \\
  In Industry Design Processes, I worked with a team of three other students to design and build a microcontroller-based project. We developed a microcontroller-based audio synthesizer capable of generating up to 6 frequencies simultaneously. Gathered requirements, prototyped solutions, designed and built a PCB, and implemented all of the firmware in a single, 10-week semester. 
  \item 
  \headerrow 
  {\textbf{Linux Kernel Driver (C)}}
  {\textbf{github.com/rattboi/blec\_dev}}
  \\
  In Linux Device Drivers, I wrote a Linux kernel driver in C for a USB-based data input/output device. It supported stable hot-plugging and removal of simultaneous devices, with separate interfaces to each device.
\end{itemize}
\vspace{1em}

\hrule
\vspace{-0.4em}
\subsection*{References}
\begin{itemize}
  \item
  \headerrow
    {\textbf{Kevin Barry}}
    {\textbf{Plant Manager, Silver Bay Seafoods}}
  \\
  \textbf{Phone: } (xxx)xxx-xxxx
  \item
  \headerrow
    {\textbf{Doug Hall}}
    {\textbf{Professor, Portland State University}}
  \\
  \textbf{Phone: } (xxx)xxx-xxxx
  \item
  \headerrow
    {\textbf{Mark Faust}}
    {\textbf{Professor, Portland State University}}
  \\
  \textbf{Phone: } (xxx)xxx-xxxx

\end{itemize}

\end{document}
